\documentclass[12pt,runningheads]{article}
\usepackage[utf8]{inputenc}

%enable apa
\usepackage[british]{babel}
\usepackage{csquotes}
\usepackage[backend=biber,style=apa]{biblatex}
\DeclareLanguageMapping{british}{british-apa}

%some additional packages
\usepackage[pdftex]{graphicx}

\addbibresource{references.bib}

\begin{document}

\title{Research Methodology in Computer and Information Science \\ \small TDT39}
\author{Magnus Settemsli Mogstad}
\date{\vspace{-5ex}}
\maketitle
\begin{figure}[h!]
	\centering
    \includegraphics[width=0.4\textwidth]{ntnu_hele_navnet_staaende_bm.png}
\end{figure}

\pagebreak


\section*{Research Plan}
The research I am currently doing is divided into two different phases, the first phase is this fall where I create an algorithm and test it then in spring I will create a system which utilizes the algorithm created in the first phase of my research, the plan will cover both phases.

\subsection*{Purpose (35 \%)}
%mangler noe føler jeg
Today people say that the earth is getting smaller by the minute because staying connected around the world is easier than before. Two of the reasons for this is because traveling is not reserved for the elite society anymore, and that the internet has taken over the world it is almost everywhere round 43 \% of the worlds population is connected to the internet[1,2]. And you can now be connected to every other person in the world which is connected to the internet just by being connecting to the internet. \newline\newline When we travel to new places we want to experience the history, the people and the culture of that place. Because of the possibility to utilize the internet and traveling we no longer plan our trips meticulously, now we just pull up our phone and use systems like TripAdvisor, stay.com etc to guide us to the places we want to see. All of these applications give you the possibility to find the different tourist attractions, hotels and restaurants in the cities you are going to visit. The motivations for this task is that none of the existing applications give you a recommendation of what attractions you should visit based on your previous tourist attraction history, personal information and your likes and dislikes. And such a system can be good because it can give previously unvisited attractions a bigger visitor number. Because people who are interested in that kind of attraction will come and visit it. \newline\newline My research aim is to create a system which can give the user a recommendation of which attraction in the area should be visited because it fits the users previous attraction pattern and his profile. The system will use implicit feedback and a user-profile with the users likes and dislikes to give recommendations to the user.\newline \textbf{RQ1: How does the previous applications give the user a suggestion of what attractions to see?}\newline
\textbf{RQ2: Is implicit feedback is better than explicit feedback?} \newline
\textbf{RQ3: Does some personal information matter when recommending attraction? children/married etc}

\subsection*{Products (20 \%)}
%Mangler noe
The main product of this research is a system which gives the user a location based recommendation of tourist attractions. The goal is to create a system which can give the user a recommendation based on a user-profile and implicit feedback from the user, the feedback is a question whether the user liked, was neutral or disliked the attraction visited. To reach this goal the we have divided the task into two different parts, the first part ... the second part is the master thesis where the system is to be implemented, tested and analyze the results from the user tests. \newline\newline Creating a recommendation system in the area of tourist attractions is a area where the amount of previous research is limited, since the area of research is so wide it is important to be able to concentrate on a few areas of attractions because of the limited amount of time. Since most the systems containing tourist attractions is closed for research purpose it does not exist any good data set containing the tourist attraction with their category and sub-category, because of this the attraction data is based on what we can get from Google. Since Google holds all the data about the attractions we only need to have data about the user and which attractions the user have visited. 

\subsection*{Process (20 \%)}
%TODO

\subsection*{Participants (10 \%)}
The participants of this research is myself my co-writer John-Olav Storvold our research supervisor Heri Ramampiaro which contributes with his experience in computer science research and his previous experience with recommendation systems utilizing implicit feedback[6].

\subsection*{Paradigm (10 \%)}
\subsection*{Presentation (5 \%)}




%text goes here

\begin{thebibliography}{9}

\bibitem{Internet Live Stats}
  	\emph{Internet Live Stats: \newline \url{http://www.internetlivestats.com/internet-users/}}, November 2015
    
\bibitem{World population}
	\emph{World population stats: \newline \url{http://www.worldometers.info/world-population/}}, November 2015
    
\bibitem{Tourism first half 2015}
	\emph{International tourism first half 2015: \newline \url{http://media.unwto.org/press-release/2015-10-29/international-tourist-arrivals-4-driven-strong-results-europe}},\newline 29 October 2015
    
\bibitem{Tourism first half 2014}
	\emph{International tourism first half 2014\newline \url{http://media.unwto.org/press-release/2014-09-15/international-tourism-5-first-half-year}}, 15 September 2014
    
\bibitem{Top 50 tourist attractions in 2014}
	\emph{Top 50 tourist attractions in 2014 \newline \url{http://www.travelandleisure.com/slideshows/worlds-most-visited-tourist-attractions?iid=sr-link1}}, November 2014

\bibitem{Fashion}
	\emph{Learning to Rank for Personalised Fashion Recommender Systems via Implicit Feedback \url{http://link.springer.com/chapter/10.1007\%2F978-3-319-13817-6_6}}, 2014
\end{thebibliography}

\end{document}
